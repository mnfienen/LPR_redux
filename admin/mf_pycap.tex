\documentclass[]{article}

\usepackage{natbib} % Supports citation styles in BibTeX.
% Title Page
\title{Comparing pycap and MODFLOW}
\author{Mike Fienen, USGS Upper Midwest Water Science Center}


\begin{document}
\maketitle

\section{Governing equations for MODFLOW}
The governing equation for groundwater flow, as implemented in MODFLOW and summarized from \citep{mf6a}, is based on Darcy's Law:
\begin{equation}
\mathbf{q}=-\mathbf{K}\nabla h=-\left(\begin{array}{ccc}
K_{xx} & 0 & 0\\
0 & K_{yy} & 0\\
0 & 0 & K_{zz}
\end{array}\right)
\nabla h
\end{equation}

where $\mathbf{q}$ is specific discharge, $\mathbf{K}$ is the hydraulic conductivity tensor, here only defined in the principal $x$, $y$, and $z$ directions, and $h$ is potentiometric head. 
Expressing this as a water balance on a small control volume (for instance, a single model cell) this can be expressed as a partial differential equation

\begin{equation}
\frac{\partial}{\partial x}\left(K_{xx}\frac{\partial h}{\partial x}\right)+\frac{\partial}{\partial y}\left(K_{yy}\frac{\partial h}{\partial y}\right)+\frac{\partial}{\partial z}\left(K_{zz}\frac{\partial h}{\partial z}\right)+\sum_{s}Q_{s}'=SS\frac{\partial h}{\partial t}
\end{equation}
where $\sum_{s}Q_{s}'$ is the sum of all sources and sinks (e.g. boundary conditions) acting on a cell, with the convention of negative sign denoting flow out of the control volume, and positive sign denoting flow into the control volume. $SS$ is the specific storage, and $t$ is time.

This control volume is applied to a three-dimensional grid of cells where $Q_{s}'$ represents both external sources and sinks representing boundary conditions and also the flow between adjacent model cells. 

To evaluate streamflow depletion, the groundwater system must be simulated with MODFLOW by first creating a grid that represents the geometry of an area, supplying values for properties such as $K_{xx}$, $K_{yy}$, $K_{zz}$, and $SS$ and boundary conditions (including wells, streams, and others). 

\section{Governing equations for pycap}
The pycap software supports multiple solutions for calculating the depletion of streamflow due to pumping in a well. The solution used in this class exercise is the Glover equation \citep{Glover_1954}.
\begin{equation}
s = Q_w \times \textrm{erfc} \left( \frac{d}{\sqrt{4\frac{T}{S}t}} \right)
\end{equation}
where $s$ is streamflow depletion, $Q_w$ is the well pumping rate, $\textrm{erfc}$ is the
complementary error function, $d$ is the minimum distance
from the well to the stream, $T$ is aquifer
transmissivity, $S$ is aquifer storage, and $t$ is time from starting of pumping. 

\section{Questions to consider}
\begin{enumerate}
\item In the Little Plover River MODFLOW model, what kinds of boundary conditions are represented by the $Q'_s$ terms? How do these interact with streamflow depletion?
\item What terms are common to the two formulations? What terms are distinct to one or the other?
\item Based on the differences in the formulations, do you expect them to result in the same calculations of depletion?
\item How is distance considered in the two formulations?
\end{enumerate}

 \bibliographystyle{apalike} 
 \bibliography{refs} 
\end{document}